% Created 2025-01-19 Sun 00:17
% Intended LaTeX compiler: pdflatex
\documentclass[11pt]{article}
\usepackage[utf8]{inputenc}
\usepackage[T1]{fontenc}
\usepackage{graphicx}
\usepackage{longtable}
\usepackage{wrapfig}
\usepackage{rotating}
\usepackage[normalem]{ulem}
\usepackage{amsmath}
\usepackage{amssymb}
\usepackage{capt-of}
\usepackage{hyperref}
\author{Akilan}
\date{\today}
\title{}
\hypersetup{
 pdfauthor={Akilan},
 pdftitle={},
 pdfkeywords={},
 pdfsubject={},
 pdfcreator={Emacs 29.1 (Org mode 9.6.6)}, 
 pdflang={English}}
\begin{document}

\tableofcontents

\begin{center}
\includegraphics[width=.9\linewidth]{/Users/akilan/.cache/org-persist/3a/9e0d0c-d34d-4242-92ac-49f332af6af0-8b003edff282ad2da5830ec63041260a.png}
\end{center}]]

The main aim of this project was to create a custom peer to peer
network. The user acting as the client has total flexibility on how to
batch the tasks and the user acting as the server has complete
flexibility on tracking the container's usages and killing the
containers at any point of time.

\subsection{Latest tutorial}
\label{sec:org9129ee1}
\href{https://www.youtube.com/watch?v=OMwCpedu5cs\%22}{\includegraphics[width=.9\linewidth]{/Users/akilan/.cache/org-persist/17/01b895-4c8f-4c97-8b7a-76bde31f87cd-8dfdb3583a175929dac0d44b86acf5b3.jpg}}

\section{Table of contents in the Docs folder}
\label{sec:org881a67e}
\begin{enumerate}
\item \href{Docs/Introduction.org}{Introduction}
\item \href{Docs/Installation.org}{Installation}
\item \href{Docs/Abstractions.org}{Abstractions}
\item \href{Docs/Implementation.org}{Implementation}
\begin{enumerate}
\item \href{Docs/ClientImplementation.org}{Client Module}
\item \href{Docs/P2PImplementation.org}{P2P Module}
\item \href{Docs/ServerImplementation.org}{Server Module}
\item \href{Docs/ConfigImplementation.org}{Config Module}
\item \href{Docs/CliImplementation.org}{Cli Module}
\item \href{Docs/PluginImplementation.org}{Plugin Module}
\end{enumerate}
\item Language bindings
\begin{enumerate}
\item \href{Docs/haskell/}{Haskell}
\end{enumerate}
\end{enumerate}

\subsection{Introduction}
\label{sec:org00598fd}
This project aims to create a peer to peer (p2p) network, where a user
can use the p2p network to act as a client (i.e sending tasks) or the
server (i.e executing the tasks). A prototype application will be
developed, which comes bundled with a p2p module and possible to execute
docker containers or virtual environments across selected nodes.

\subsubsection{Objectives}
\label{sec:org385b937}
\begin{itemize}
\item Background review on peer to peer network, virtual environments,
decentralized rendering tools and tools to batch any sort of tasks.
\item Creating p2p network
\item Server to create a containerised environment
\item The client node to run tasks on Server containerised node
\end{itemize}

\href{Docs/Introduction.md}{Read more on the introduction}

\subsection{Extend your application with P2PRC}
\label{sec:orgcab50f2}
\begin{verbatim}
package main

import (
    "fmt"
    "github.com/Akilan1999/p2p-rendering-computation/abstractions"
)

func main() {
    _, err := abstractions.Init(nil)
    if err != nil {
        fmt.Println(err)
        return
    }

    // start p2prc
    _, err = abstractions.Start()
    if err != nil {
        fmt.Println(err)
        return
    }

    // Run server till termination
    for {

    }
}
\end{verbatim}

\subsubsection{Export once this is added export P2PRC as environment paths}
\label{sec:orgd4e7f7c}
\begin{verbatim}
export P2PRC=<PROJECT PATH>
export PATH=<PROJECT PATH>:${PATH}
\end{verbatim}

\href{Docs/Abstractions.md}{Read more} \ldots{}

\subsection{Installation from source}
\label{sec:org6f12be0}
\begin{enumerate}
\item Ensure the Go compiler is installed

\begin{verbatim}
go version
\end{verbatim}

\item Ensure docker is installed (Should run without sudo)

\begin{verbatim}
docker ps
\end{verbatim}

\item Clone this repository

\begin{verbatim}
git clone https://github.com/Akilan1999/p2p-rendering-computation
\end{verbatim}

\item Install and build the project

\begin{verbatim}
make install
\end{verbatim}

\item If you look closely you will get outputs such as:
\texttt{// Add them to your .bashrc file        export P2PRC=/<path>/p2p-rendering-computation  export PATH=/<path>/p2p-rendering-computation:\$\{PATH\}}

\setcounter{enumi}{4}
\item Test if it works

\begin{verbatim}
p2prc -h
\end{verbatim}

or

\begin{verbatim}
./p2prc -h
\end{verbatim}

\href{Docs/Installation.md}{Read more on the installation and
usage}
\end{enumerate}

\subsection{Design Architecture}
\label{sec:org2e297a7}
The design architecture was inspired and based on the linux kernel
design. The project is segmented into various modules. Each module is
responsible for certain tasks in the project. The modules are highly
dependent on each other hence the entire codebase can be considered as a
huge monolithic chuck which acts as its own library

\href{Docs/DesignArchtectureIntro.md}{Read more on the Design
Architecture}

\subsection{Implementation}
\label{sec:orgd4b0943}
The programming language used for this project was Golang. The reason Go
lang was chosen was because it is a compiled language. The entire
codebase is just a single binary file. When distributing to other linux
distributing the only requirement would be the binary file to run the
code. It is easy to write independant modules and be monolithic at the
sametime using Go. Using Go.mod makes it very easy to handle external
libraries and modularise code. The go.mod name for the project is
git.sr.ht/\textasciitilde{}akilan1999/p2p-rendering-computation.

\href{Docs/Implementation.md}{Read more on the Implementation}

\subsection{Find out more}
\label{sec:orge8fec0a}
As we are working on the open source project p2prc (i.e p2p network
designed for computation).If you are interested in participating as a
contributor or just providing feedback on new features to build or even
just curious about new features added to the project. We have decided to
create a discord group.
\end{document}